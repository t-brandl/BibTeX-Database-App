\documentclass[pdftex,a4paper,halfparskip]{scrartcl}
\usepackage{ngerman}
\usepackage[latin1]{inputenc} %Zum erkennen von Umlauten

\usepackage[T1]{fontenc} %Font-Encodierung wird auf das T1-Format mit bis zu 256 (anstelle 128 im Default Fontencoder) umgeschaltet

\usepackage{hyperref}
\usepackage{graphicx}
\usepackage{color}

\title{ Deeplearning - Anforderungen an die Hardware} %Definition des Titels
\author{Tobias Brandl}	%Definition des Autors

%Beginn des Dokumentkörpers
\begin{document}

%-----------------------------------------------------------------------------
\maketitle	%Titel wird erstellt

\begin{abstract}
Zuerst wird erl\"autert, wie Deeplearning funktioniert und welche Anforderungen es an heutige Hardware besitzt.
Im Anschluss werden die speziellen Anforderungen an die CPU und GPU erarbeitet, sowie deren Aufgaben erkl\"art.
\end{abstract}


\tableofcontents

\section{Einf\"uhrung und Motivation}
Motivation

\section{Zusammenfassung}
Summary

\section{Deep Learning}

\subsection{Funktionsweise}
Social Media \cite{lecun-nature-15}.
Kon-\\ventionelle Machine

\begin{itemize}
        \item Lokale Verbindungen
        \item Geteilte Gewichte
        \item Pooling
        \item Viele Schichten
\end{itemize}

\subsection{Anforderungen des Deeplearning}
ReLU: $f(z) = max(0, z)$ \cite{lecun-nature-15}.
k\"urzen \cite{lecun-nature-15}.
Netz \cite{benchmark-article} .


\section{Aufgaben der CPU}
Anwendungsfall berechnen \cite{intel-outperform}. Nervensystem Insekt

\section{Aufgaben der GPU}
Matrixoperationen \cite{gpu-mem-mngmnt}.
ewqofbwgigfbeligv edswbfedkfgbek \cite{nvidia-tensor}.
Anwendungen \cite{nvidia-tensor}.
Tensor-Core Technologie.\cite{nvidia-tensor}
Taktzyklus.\cite{nvidia-tensor}

\pagebreak
\section{Ausblick}
 Toleranz der Trainingsdauer \cite{imagenet-class}.

\bibliographystyle{abbrv}

\bibliography{paperliteratur}

\end{document}